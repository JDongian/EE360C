\documentclass{article}
\usepackage{amssymb}
\usepackage{amsmath}
\usepackage{centernot}
\setlength{\parindent}{0in}

\usepackage{color}
\usepackage{listings}
\lstset{ %
    language=C,                % choose the language of the code
    basicstyle=\footnotesize,       % the size of the fonts that are used for the code
    numbers=left,                   % where to put the line-numbers
    numberstyle=\footnotesize,      % the size of the fonts that are used for the line-numbers
    stepnumber=1,                   % the step between two line-numbers. If it is 1 each line will be numbered
    numbersep=5pt,                  % how far the line-numbers are from the code
    backgroundcolor=\color{white},  % choose the background color. You must add \usepackage{color}
    showspaces=false,               % show spaces adding particular underscores
    showstringspaces=false,         % underline spaces within strings
    showtabs=false,                 % show tabs within strings adding particular underscores
    frame=single,           % adds a frame around the code
    tabsize=2,          % sets default tabsize to 2 spaces
    captionpos=b,           % sets the caption-position to bottom
    breaklines=true,        % sets automatic line breaking
    breakatwhitespace=false,    % sets if automatic breaks should only happen at whitespace
    escapeinside={\%*}{*)}          % if you want to add a comment within your code
}

\begin{document}

\title{EE360C: Lab 1}
\author{Joshua Dong}
\date{\today}
\maketitle

\subsection{Existence Proof}
We will show that a weakly stable matching exists using the Gale-Shapely
algorithm. First we show that the algorithm terminates and produces a
matching S. Then we will show that matching is weakly stable.
\\\\
All apartments are either paired or free. If, by way of contradiction, a
tenant applies to all apartments but is rejected by all of them, then all the
apartments must be paired. Since there are an equal number of tenants and
apartments and no tenants has two apartments, this would imply all tenants
have found an apartment, a contradiction. Thus the Gale-Shapely algorithm
always produces a matching.
\\\\
Each tenant proposes one every round of matching, and never applies to the
same apartment twice. Therefore there cannot be more than $n^2$ rounds, where
$n$ is the number of apartments or tenants. Then the algorithm terminates.
\\\\
On termination of the algorithm, let the resultant matching be S. If a tenant
prefers another apartment to their current apartment, then they must have been
rejected by that apartment previously. If they were rejected by an apartment,
then that apartment must have been paired with a tenant the landlord preferred
more. Then that apartment does not prefer the tenant over its current tenant.
Since this argument applies to all tenants, we conclude that no unstable
matchings exist. Then the matching is weakly stable.


\newpage
\subsection{Algorithm}
We provide the algorithm used in the previous section:

\begin{lstlisting}
function stableMatching (tenant[] Tenants, apartment[] Apartments):
    initialize all tenants and apartments to free
    while there exists a free tenant T:
        A_t = top choice apartment for T not yet applied to
        if A_t is free:
            (T, A_t) become a match
        else:
            let T' be the tenant of A_t in the preexisting matching (T', A_t)
            if A_t prefers T to T':
                T' becomes free
                (T, A_t) become engaged 
            else:
                (T', A_t) remain engaged
                A_t is removed from the list of choices for T
\end{lstlisting}


\subsection{Correctness}
Each tenant proposes at most once every round of matching, and never applies
to the same apartment twice. Therefore there cannot be more than $n^2$ rounds,
where $n$ is the number of apartments or tenants. Then the algorithm
terminates since $n^2$ is a finite upper bound.
\\\\
We follow the same steps as taken in Existence Proof section. Since our
algorithm is the tenant-optimal Gale-Shapely algorithm, we know that the
resultant matching is weakly stable.
\\\\
Our program terminates and always produces correct output. Therefore, it is 
totally correct.


\subsection{Runtime Complexity}
We already showed that our program is bounded by $n^2$ where $n$ is the number
of apartments or tenants. Then our program is $O(n^2)$. We also know that our
algorithm is also bounded below by $n$ since each tenant must find a match and
at minimum only applies to one apartment. Thus our proposal accurately
captures the runtime complexity of our program.

\subsection{Brute Force}
In a brute force solution, we would produce all matchings and filter for
stable matchings. There are $n^2$ matchings and this would always take
$n^2$ checks where $n$ is the number of apartments or tenants. Thus the
runtime complexity for the brute force algorithm is $O(n^2)$.

\end{document}
